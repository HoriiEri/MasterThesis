\documentclass[a4j,10pt]{jsarticle}
%
\usepackage[top=10mm,left=15mm,right=15mm, bottom=15mm]{geometry}
\title{修士論文 質問回答書}
\author{慶應義塾大学 理工学研究科 開放環境科学専攻 修士二年生\\学籍番号81418721 池田 泰成}
\date{2015年2月}

\begin{document}
\begin{flushright}
平成28年1月19日
\end{flushright}


\noindent
慶應義塾大学 理工学研究科\\
開放環境科学専攻 情報工学専修 御中

\vspace{3cm}
\begin{center}
{\Huge 修士論文発表における質問に対する回答文}
\end{center}

\vspace{3cm}
\noindent
{\Large 著者学籍番号:81418721\\
\\
題目:\\
頭髪状物体の高品質な影付けのための\\オブジェクトスペースアンチエイリアシング手法\\}

\vspace{3cm}
\noindent
{\Large 拝啓\\
時下ますますご清祥のこととお慶び申し上げます.\\
この度は,私の修士論文発表において貴重なコメントを賜りまして誠にありがとうございました.頂戴したコメントに従い,以下のような方針に基づいて改良を加えましたので,ご確認を何卒よろしくお願い申し上げます.\\}


\begin{flushright}
{\Large 敬具




池田 泰成}
\end{flushright}

\newpage
\noindent
\textbf{\underline{質問1:現在のアンチエイリアシング品質が十分であると答える根拠は何か.(岡田先生)}}\par
定量的なエイリアシング量の測定実験において,提案手法はパラメータの設定により強くエイリアシングを抑制できることが証明されました.適切なパラメータを設定した場合には,ほぼこれ以上のエイリアシング抑制が不可能な状態までエイリアシングを抑制できているため,発表の場では「エイリアシングの抑制力については十分」と回答しました.しかし,数値上問題ないとしても人間の感覚でエイリアシングを感じられた場合には,別のエイリアシングを抑制できていないということになるため,対応が必要だと考えております.したがって,このご質問への回答は以下のようになります.\par
エイリアシングの定量的計測結果を見ると,パラメータの設定によってはエイリアシングの量は既存手法に比べてひじょうに少ないため現時点でエイリアシングの抑制力は十分であると考える.これ以上の対処が必要かどうかは,定性的評価を行ったうえで対応を考える必要がある.\\
\\
\textbf{\underline{質問2:高速化に関する方針はどのようなものを考えているか.(萩原先生)}}\par
この質問に関する回答といたしまして,考察した内容を本文の第5章第2節第1項の高速化の項目に以下のように文章を記載いたしました.\\
\\
「第4 章でも述べたように,提案手法はユーザの入力に対して即座の返答を返すような計算速度を保っているが,DOMs に比較して計算速度は低下している.この原因は,本来三角メッシュで表現されているオブジェクトを分解し,各メッシュを4 頂点で表現される板ポリゴンで表示しているためである.全体のポリゴン数が少ない場合にはこの点は問題とならないが,今回の実験で使用したような大きな3D モデルの場合には,計算時間の差が問題となる.また,頭髪の太さを表すthicknessを広げると個々の板ポリゴンのサイズが大きくなり,GPU でアクセスするセグメントの個数が多くなってしまう.計算速度を上げるためには,板ポリゴンの個数を減らしたり,密度の計算方法を変更してセグメントの個数が増えても計算量が増えにくいような計算方法を考える必要がある.板ポリゴンを減らすアプローチは,先行計算によりオブジェクト全体の密度分布がわかっていれば密度が十分に高い位置のポリゴン数を減らすような方法が考えられるが,頭髪はオブジェクトの形状が変わりやすく,密度の分布も変化しやすいため,先行計算により前もってオブジェクトの大まかな密度分布を計算する方法は現実的ではない.これに対して,密度の計算方法を変更する方法は実現可能性が高い.板ポリゴン上の不透明度分布はガウシアンカーネルに従っているため,一部の不透明度のみを計算し,ほかの場所に関しては線形補完して得た値であっても,実際に計算した値との誤差が小さいと考えられる.」\\
\\
\textbf{\underline{質問3:パラメータ決定方法はどのように決定していますか.その計算を自動化することはできないのでしょうか.(萩原先生)}}\par
板ポリゴンのパラメータは,$thickness$,$opacity$ともにユーザの入力に任せています.CGの作品では計算上正しい値であっても作品の都合上色を強調する,弱くするというような調整が必要となるためです.ただし調整機能は残したまま,パラメータの初期値をより適切と考えられる値に自動設定する機能はひじょうに重要です.今回考察した実現方法は以下のような方法があります.\par
\textbf{これまでのユーザ入力データから計算する方法.}この方法では,頭髪状物体の形状と,ユーザが入力した$thickness$,$opacity$を記録しておきます.もし新しい物体が入力された場合には,これまでの物体形状の中で似た形状のものを参照または複数のデータを合成し,$thickness$,$opacity$を決定します.データベースを使う方法は近年のCG研究でもよく見られる方法であるため,実現可能性は高いと考えています.\par
\textbf{先行計算によるパラメータ予測.}計算を開始する前に,頭髪状物体全体の密度分布と頭髪同士の距離を先行取得しておきます.この上で,投影される影の本影と半影の割合をあらかじめ設定しておけば,そこからどの程度の密度を割り当てれば目的通りの影濃度が得られるか計算できます.また,頭髪同士の距離を知ることで,板ポリゴンの配置位置が特定できるため,どの程度板ポリゴンを拡大すればエイリアシングが抑制されるか計算できます.\\
\\
%
\textbf{\underline{質問4:ズームイン,ズームアウトで問題なくアンチエイリアシングできるのか,確認する必要はありますか.(藤代先生)}}\par
現時点では確認しておりませんが,前期博士課程卒業までにユーザによる品質評価を実施しようと考えているため,その項目の一つとしてズームイン,ズームアウトでの品質評価を加えさせていただきます.\\
\\
\begin{flushright}
以上
\end{flushright}
\end{document}