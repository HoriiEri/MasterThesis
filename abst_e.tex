\begin{center}
{\bf {\large Thesis Abstract}}

\vspace{2ex}

{\bf {\large BAND: Blowing Animation created from Natural motions synchronized with Digital music}}
\end{center}

\vspace{3ex}

\parindent 0.4 in

%音楽演奏をテーマにしたアニメーションはたくさんある.
There exists a lot of animation with the theme of music performance.
%
%テレビで放映されたアニメーションの例は,『のだめカンタービレ』,『けいおん!』,『響け!ユーフォニアム』である.
An example of animation aired on television are "Nodame Cantabile", "K-ON!", and "Hibike! Euphonium".
%
%これらのアニメーションの制作方法は,セル画や3DCGである.そして,いずれも実際の演奏者の動きに近い演奏シーンが生成されている.
Even though the production methods of these series are different, all of them focus on generating realistic movements of actual performers.
%These animation production methods are cell images or 3D CG, and each performance animation is close to actual performer's movement.
%
%楽器の輝きや形状なども忠実に再現されている.
The shine and shape of the instruments are faithfully reproduced.
%
%しかし,楽器を演奏するキャラクタの運指や身体の動きが,音楽に完全に同期されていないことがある.
However, sometimes the performer's fingering and movement are not completely synchronized with music and gives a sense of incompatibility.
%
%特に速いフレーズや,複雑なリズムを演奏するシーンで,このようなアーティファクトが起きやすい.
This type of artifact occurs especially in a scene that plays a fast phrase or a complex rhythm.
%
%また,表情からも不自然さを感じることがある.
Also, occasionally facial expressions in animation series are unnatural.
%
%これらの違和感を解消するためには,身体の動きを1つずつ音階やリズムなどに,手動で合わせる必要がある.
In order to eliminate these discomfort, 
it is necessary to manually synchronize the movement of the body one by one with the scale, rhythm, and so on.
%
%この方法は効率が悪く,多くの時間と労力が必要となる.
This method is inefficient, so much time and effort are required.\\
\indent
%
%上記の課題を解決するため,鍵盤楽器や弦楽器を演奏する演奏者のアニメーションを,音源から自動生成する研究が存在する.
In order to solve these issues, there are works to automatically generate animation of performers playing keyboard instruments and stringed instruments from sound sources.
%
%しかし,管楽器を対象とした研究は存在しない.
However, as far as we know, there are no works targeting wind instruments.
%
%そこで本研究では,管楽器を演奏するキャラクタの吹奏アニメーションを,音源から自動生成することを目指した.
Therefore, in this work, we aimed to automatically generate blowing animation of characters playing wind instruments from sound sources.
%
%アニメータの作業支援を目的とするため,自動生成の流れは実際の演奏アニメーション制作フローに沿わせた.
For the purpose of animator's work assistance, Animation was generated in accordance with the actual animation production flow of musical performance.
%
%より具体的には,楽曲は電子楽器を用いて,MIDI音源として生成する.
More specifically, the sound sources is generated as a MIDI sound source using an electronic musical instrument.
%
%次に,作成した楽曲を解析することにより,音楽の情報を得る.
Next, we analyze the generated sound source to obtain music information.
%
%最後に,得られた情報をキャラクタの身体の動きや表情,そして管楽器に適用することにより,音源に同期した自然な吹奏アニメーションを実現した.
Finally, by applying the obtained information to the character's movement, facial expression, 
and wind instruments, we generated natural blowing animation synchronized with the sound source.\\
\indent
%
%自動生成したアニメーションについて評価を行った結果,提案手法により,自然なアニメーションを自動的に生成できることを確認した.
Evaluation on automatically generated animation revealed that the proposed method can gerenate natural animation automatically.

\vspace{4ex}

\noindent
{\bf Keywords}

\noindent
Animation; wind orchestra; brass instrument; automatically generation.

\parindent 1zw