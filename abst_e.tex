\begin{center}
{\bf {\large Thesis Abstract}}

\vspace{2ex}

{\bf {\large Automatic Generation of Blowing Animation Taking Fingering and Expression Synchronized with Digital Music into Account}}
\end{center}

\vspace{3ex}

\parindent 0.4 in
There are known many animation works with the theme of music performance. Representative televised example of these include ``Nodame Cantabile", ``K-ON!", and ``Hibike! Euphonium", all of which aim at commonly generating realistic motion of performers as well as faithfully reproducing the shape and appearance of the instruments though the underlying production method varies from each other. However, the performer's fingering and motions are sometimes not completely synchronized with music, and thus giving a sense of incompatibility. This type of artifact occurs especially in a scene that he/she plays a fast phrase or in a complex rhythm. Also, facial expressions in animation series occasionally look unnatural. In order to eliminate these kinds of discomfort, it is necessary to manually synchronize the movement of the body one by one with the scale, rhythm, and so on. This work is inefficient because tedious work and much effort are required.\par
%
In order to address these issues, there are known works to automatically generate animations of performers playing keyboard instruments and stringed instruments from sound sources. However, to the best of the author's knowledge, there are no previous works targeting at wind instruments. Therefore, in this thesis research, we aimed to automatically generate blowing animation of characters playing wind instruments from sound sources. For the purpose of animator's work assistance, animation was generated in accordance with the actual animation production flow of musical performance. More specifically, the sound source was generated as a MIDI sound source using an electronic musical instrument. Next, we analyzed the generated sound source to obtain music information. Finally, by applying the obtained information to the character's movement, facial expression, and wind instruments, we generated natural blowing animation synchronized with the sound source.\par
%
Evaluation on the resultant animation by third parties empirically proved that the proposed method can generate natural animation automatically.
\vspace{4ex}

\noindent
{\bf Keywords}

\noindent
Animation; wind orchestra; brass instrument; and automatically generation.

\parindent 1zw