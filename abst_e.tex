\begin{center}
{\bf {\large Thesis Abstract}}

\vspace{2ex}

{\bf {\large An Object-Space Anti-Aliasing Method\\for High-Quality Shadowing of Hair-Shaped Objects}}
\end{center}

\vspace{3ex}

\parindent 0.4 in

%音楽演奏をテーマにしたアニメーションはたくさんある.
There are a lot of animation with the theme of music performance.
%
%テレビで放映されたアニメーションの例は,『のだめカンタービレ』,『けいおん!』,『響け!ユーフォニアム』である.
An example of animation aired on television are "Nodame Cantabile", "K-ON!", and "Hibike! Euphonium".
%
%これらのアニメーションの制作方法は,セル画や3DCGである.そして,いずれも実際の演奏者の動きに近い演奏シーンが生成されている.
These animation production methods are cell images or 3D CG, and each performance animation is close to actual performer's movement.
%
%楽器の輝きや形状なども忠実に再現されている.
The glow and shape of the instrument are faithfully reproduced.
%
%しかし,楽器を演奏するキャラクタの運指や身体の動きに注目すると,音楽と動きが完全に同期されていないことがあり,違和感を感じる.
However, pay attention to the performer's fingering and mvoement, 
sometimes it is not completely synchronized with music and gives a sense of incompatibility.
%
%特に速いフレーズや,複雑なリズムを演奏するシーンで,このようなアーティファクトが起きやすい.
This type of artifact is occur especially in a scene that plays a fast phrase or a complex rhythm.
%
%また,表情からも不自然さを感じることがある.
Also, occasionally the facial expression is unnatural.
%
%これらの違和感や不自然さを解消するためには,身体の動きを1つずつ音階やリズムなどに,手動で合わせる必要がある.
In order to eliminate these discomfort and unnaturalness, 
it is necessary to manually synchronize the movement of the body one by one to the scale, rhythm, and so on.
%
%この方法は効率が悪く,多くの時間と労力が必要となる.
This method is inefficient, so much time and effort are required.\\
\indent
%
%上記の課題を解決するため,鍵盤楽器や弦楽器を演奏する演奏者のアニメーションを,音源から自動生成する研究が存在する.
In order to solve this problem, there are works to automatically generate animations of performers playing keyboard instruments and stringed instruments from sound sources.
%
%しかし,管楽器を対象とした研究は存在しない.
However, there are no works targeting wind instruments.
%
%そこで本論文では,管楽器を演奏するキャラクタの吹奏アニメーションを,音源から自動生成することを目指す.
Therefore, in this paper, we aim to automatically generate blowing animation of characters playing wind instruments from sound sources.
%
%より具体的には,実際の演奏アニメーション制作フローに沿わせるため,楽曲は電子楽器を用いて,MIDI音源として生成する.
More specifically, in order to follow the actual musical performance animation production flow, music is generated as a MIDI sound source using an electronic musical instrument.
%
%次に,作成した楽曲を解析することにより,音楽の情報を得る.
Next, by analyzing the generated music, we obtain music information.
%
%
Finally, by applying the obtained information to the character's movement, facial expression, 
and wind instruments, natural blowing animation synchronized with the sound source is generated.
%
%また,提案手法の対象ユーザはアニメータである.
Also, the target user of the proposed method is an animator.\\
\indent
%
%自動生成したアニメーションについて評価を行った結果,提案手法により,自然なアニメーションを自動的に生成できることを確認した.
Evaluation on automatically generated animation revealed that the proposed method can gerenate natural animation automatically.

\vspace{4ex}

\noindent
{\bf Keywords}

\noindent
Animation; wind orchestra; brass instrument; automatically generation.

\parindent 1zw