\begin{center}
{\bf {\large 論文要旨}}
\end{center}

\vspace{3ex}
音楽演奏を題材としたアニメーションは多く存在する.
そこでは,実際に演奏する演奏者の身体の動きに近い演奏シーンが生成されている.
しかし,楽器を演奏するキャラクタの運指に注目すると,音楽と指の動きが完全に同期されていないことがあり,強い違和感を感じる.
また,完全に同期されているアニメーションを作成するには,多くの労力が必要となる.
そこで本研究では,管楽器を演奏するキャラクタの吹奏アニメーションを,音楽に容易に同期させるため,音源から自動的に運指のアニメーションを生成することを目指す.
より具体的には,ウインドシンセサイザを用いて作成したMIDI音源を解析することにより,運指が音源に同期した,自然な吹奏アニメーションを実現する.

近年,コンピュータの処理速度が向上したことで,高品質な画像の高速な描画が可能となった.
%
この影響により,ゲームや映像をユーザが完成版に近い品質で編集できる技術が発達している.
%
このようなCG技術の台頭やITの発達で,最新のコンピュータ上に仮想の社会や人間をつくる技術が研究者やCG関係者の注目を集めている.
%
しかし人間のパーツの多くは目や肌,髪の毛のようにやわらかい,小さい,\emph{半透明}といった特徴をもつパーツで構成され,コンピュータには簡単に描画できないものが多い.
%
特に\emph{頭髪}は,人間にとって重要なパーツでありながら,その複雑さが原因で高速な描画が難しい物体として知られている.
%
また頭髪状物体が落とす影も同様に,一般的な描画手法を使うと\emph{ちらつき}やありえない濃度の影が生じるため,計算が難しい描画対象として知られている.\par

頭髪は半透明で,かつ細長い円筒の集合であり,このような幾何形状をもつ物体は頭髪の形状,位置や視点,照明に変化がある場合,影を描画する際にちらつきが生じる.
%
このちらつきのうち,頭髪が半透明で遮蔽判定が難しいことによるちらつきは,影を描く場所に光が届くまでに通過する\emph{頭髪密度の積算値}を使い影を描くことで解決できる.
%
一方,髪が細すぎてコンピュータが髪を認識できないことによるちらつきは,投影後にちらつきが生じている部分と周囲の色を平均化して抑制されてきた.
%
しかしこの平均化処理は,結果の品質が投影後の単位面積あたりの画素数に依存してしまう.\par
そこで本論文では,頭髪の密度を積算する既存手法を拡張して,大きさが可変で不透明度分布をもつ正方形小板で\emph{頭髪の三次元形状を近似}することにより,結果の品質が画素数に依存しない\emph{アンチエイリアシング}手法を提案する.
%
これに加えて既存研究では積極的に行われてこなかった,出力画像の定量的な品質評価をするために新たな評価方法を作成した.
%
この評価方法を使用して既存手法と提案手法の出力画像の品質を比較した結果,提案手法は計算速度を大きく減少させることなく効果的にちらつきを抑制できることを確認した.

\vspace{4ex}

\noindent
\textgt{キーワード}

\noindent
アニメーション,吹奏楽,金管楽器,自動生成.