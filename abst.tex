\begin{center}
{\bf {\large 論文要旨}}
\end{center}

\vspace{3ex}
音楽演奏を題材としたアニメーションは多く存在する.
テレビで放映されたアニメーションの例を挙げると,『のだめカンタービレ』,『けいおん!』,『響け!ユーフォニアム』が相当する.
これらのアニメーションは,セル画であったり3DCGであったり,製作方法はさまざまであるが,いずれも実際に演奏する演奏者の身体の動きに近い演奏アニメーションが生成されている.
楽器の輝きや形状なども忠実に再現されている.
%
しかし,楽器を演奏するキャラクタの運指や身体の動きが,音楽に完全に同期されていないことがある.
特に速いフレーズや,複雑なリズムを演奏するシーンで,このようなアーティファクトが起きやすい.
また,表情からも不自然さを感じることがある.
これらの違和感を解消するためには,身体の動きを1つずつ音階やリズムなどに手動で合わせる必要がある.
この方法は効率が悪く,多くの時間と労力が必要となる.\\
%
\indent
上記の課題を解決するため,鍵盤楽器や弦楽器を演奏する演奏者のアニメーションを,音源から自動生成する研究が存在する.
しかし,管楽器を対象とした研究は,我々の知るかぎりでは存在しない.
そこで本研究では,管楽器を演奏するキャラクタの吹奏アニメーションを,音源から自動生成することを目指した.
アニメータの作業支援を目的とするため,自動生成の流れは実際の演奏アニメーション制作フローに沿わせた.
より具体的には,楽曲は電子楽器を用いてMIDI音源として生成する.
次に,作成した楽曲を解析することにより,吹奏の情報を得る.
最後に,得られた情報をキャラクタの身体の動きや表情,そして管楽器に適用することにより,音源に同期した自然な吹奏アニメーションを実現した.
自動生成したアニメーションについて評価を行った結果,提案手法により,自然なアニメーションを自動的に生成できることを確認した.

\vspace{4ex}

\noindent
\textgt{キーワード}

\noindent
アニメーション,吹奏楽,金管楽器,自動生成.