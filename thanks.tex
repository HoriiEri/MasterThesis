\chapter*{謝辞}
\label{chap:thanks}
%
\addcontentsline{toc}{chapter}{謝辞} % 目次へ追加
%
本論文の執筆にあたり,ひじょうに多くのアドバイスを頂いたばかりでなく,さまざまな相談も聞いていただきました,慶應義塾大学理工学部情報工学科の藤代一成教授に深く感謝いたします.
特に,修士1年生の途中で研究テーマを変更するか否か迷っていた際に,多大な後押しをしていただきました.
藤代教授の後押しがあったからこそ,後悔のない研究室生活を送ることができました.
藤代研究室の一員として受け入れていただけたことを,改めて嬉しく思います.
本当にありがとうございました.\\
\par
本論文の執筆にあたり,貴重なご意見を頂きました,慶應義塾大学理工学部情報工学科の斎藤英雄教授,杉本麻樹准教授,杉浦裕太助教に深く感謝いたします.
お陰様で,広い視野を持って本論文を執筆することができました.\\
\par
本研究を進めるにあたり,実装面において多くのアドバイスを頂いただけでなく,コンピュータグラフィクス分野におけるさまざまな課題に出会う機会や,実際に業務に携わる機会を頂きました,株式会社デジタル・フロンティア開発部の皆様に深く感謝いたします.\\
\par
本研究を進めるにあたり,時には楽しく話し合い,時には真面目に議論を行い,共に助け合った研究室の皆様に深く感謝いたします.
特に,共同で研究を進めてくれた研究室学士4年の武内氏は,吹奏楽の経験がないにも関わらず,何度もディスカッションを重ねることにより知識を取得し,重心移動や合図などのモーションを作ってくれました.
また同期の皆様は,時には優しく,時には厳しく,私の研究室生活を支えてくれました.このメンバーでなければ,ここまで楽しく充実した研究室生活を送ることはできませんでした.\\
\par
最後に,本研究の評価アンケートにご協力いただきました,藤代研究室の皆様,株式会社デジタル・フロンティア開発部の皆様,お茶の水女子大学伊藤研究室の皆様,株式会社スクウェア・エニックス2018年新卒の皆様,東京個別指導学院高円寺教室の講師の皆様,應援指導部の皆様,慶應義塾大学塾生会館受付アルバイトの皆様,高校時代のクラスメイトの皆様に,厚く御礼申し上げます.