\chapter{序論}
\label{chap:intro}

本章では,本研究の研究背景と研究目的について説明する.
\ref{sec:background}節で研究背景,
\ref{sec:purpose}節で研究目的,
\ref{sec:structure}節で本論文の構成について述べる.

\section{研究背景}\label{sec:background}
\indent
音楽演奏を題材としたアニメーションは多く存在する.
テレビで放映されたアニメーションの例を挙げると,『のだめカンタービレ』,『けいおん!』,『響け!ユーフォニアム』が相当する.
これらのアニメーションは,セル画であったり3DCGであったり,製作方法がさまざまであるが,いずれも実際に演奏する演奏者の身体の動きに近い演奏シーンが生成されている.
楽器の輝きや形状なども忠実に再現されており,評判が高い.\\
\indent
一方,楽器を演奏するキャラクタの運指や身体の動きに注目すると,音楽と動きが完全に同期されていないことがあり,強い違和感を感じる.
特に速いフレーズや,複雑なリズムを演奏するシーンで,このようなアーティファクトが起きやすい.
また,表情からも不自然さを感じることがある.
しかし,1つ1つの音と動きや表情を合わせるには,多くの労力が必要となる.

\section{研究目的}\label{sec:purpose}
\indent
\ref{sec:background}節で述べた課題を解消するため,本研究では,管楽器を演奏するキャラクタの吹奏アニメーションを,音源から自動的に生成することを目指す.
より具体的には,実際の演奏アニメーション制作フローに沿わせるため,楽曲は楽器(ウインドシンセサイザ)を用いて,MIDI音源として生成する.
そして,作成した楽曲を解析することにより,吹奏の情報を得る.
得られた情報をキャラクタの身体の動きや表情,そして管楽器に適用することにより,音源に同期した自然な吹奏アニメーションを実現する.

\section{本論文の構成}\label{sec:structure}
本論文は次章以降,以下のように構成される.\\
\indent
次章では関連研究を説明する.
第3章では本手法の仕組みを説明する.
そして,第4章では自動生成結果を述べ,この結果に対する評価をまとめる.
最後に第5章で結論を述べ,今後の課題に言及する.\\
\indent
なお,本研究の成果はVisual Computing/グラフィクスとCAD合同シンポジウム2017にてポスター発表\cite{vc}を行った.そして,映像表現・芸術科学フォーラム2018(2018年3月),Cyberworlds(2018年8月)において発表を行う予定である.